\documentclass{article}
\setlength{\parindent}{0in}
\begin{document}
\title{Tor Metrics - Rserve}
\author{by Kevin Berry \texttt{<kevin.berry@villanova.edu>}}
\maketitle
\section{Overview}
Rserve is a TCP/IP interface which allows other tools and languages to use
the facilities of the R language. In our case, Java, Tomcat, Postgres and
other parts of Metrics work with Rserve to generate graphs on-demand. For
more information about the Metrics website, see the sister repository
\emph{metrics-db}. Here, we will cover how to install R, Rserve, the R
Postgres driver, and ggplot2. For more information about the Postgres
setup, see manual.pdf. The database should be set up before continuing
this.

\section{Architecture}
See the \emph{rserve} directory for the start/stop scripts and
config. The graph code is all pre-loaded when Rserve starts, so, if any
changes are made to the graph code, Rserve must be restarted. The database
name, user, and password can be configured in \emph{rserve-init.R}, as well
as the pre-loaded libraries. Rserve forks itself upon connection, so R code
can be pre-loaded to speed things up.

\section{Setup}
\subsection{Installing R}
Before we get started, we need to have R installed. We need to have the R
dev package installed so we can use the add-ons.

\begin{verbatim}
$ sudo apt-get install r-base-dev
\end{verbatim}

\subsection{Installing and testing Rserve}
There are a few different ways to install Rserve. However, the easiest and
most direct way to install it is through R's built-in package manager and
package network, CRAN (\emph{The Comprehensive R Archive Network -
http://cran.r-project.org}). Unfortunately, Rserve isn't packaged currently
for many Linux distributions, so it requires a bit manual configuration and
administration.
\\

R needs to be started as root so its build-in package manager can access
the file system to install its own packages. Select the mirror through the
Tcl/tk or command line interface and it should install.

\begin{verbatim}
$ sudo R
> install.packages("Rserve")
\end{verbatim}

We want to start a bare server and see if it works correctly.

\begin{verbatim}
$ R CMD Rserve
\end{verbatim}

Now, test it with the built-in R connector.

\begin{verbatim}
$ R
> library(Rserve)
> c <- RSconnect()
> RSshutdown(c)
\end{verbatim}

If this worked, the server is listening, so it installed and started
correctly.

\subsection{Installing ggplot2}
ggplot2 is the second necessary R package that we need for Metrics.

\begin{verbatim}
$ R
> install.packages("ggplot2")
\end{verbatim}

\subsection{Installing and testing the R PostgresSQL driver}
The Postgres driver installs similarly to the other R packages.

\begin{verbatim}
$ R
> install.packages("RPostgreSQL")
\end{verbatim}

First, make sure Postgres is started and configured correctly according to
the Metrics specifications (see manual.pdf).
\\

Start the R console, load the driver, and connect to the database. The
database user, password may need to be changed.
\begin{verbatim}
$ R
> library(RPostgreSQL)
> drv <- dbDriver("PostgreSQL")
> con <- dbConnect(drv, user="ernie", password="", dbname="tordir")
> dbDisconnect(con)
> dbUnloadDriver(drv)
\end{verbatim}

\section{Administrating Rserve}
Since Rserve is not standardly packaged, a few things must be done to
ensure it runs smoothly and securely. We need to adjust permissions, add
users, and modify groups so it works nicely with Tomcat. Feel free to do
this differenty according to your system's requirements.

\subsection{Adding users and groups}
We'll add the user 'rserve' without a shell and no home directory.

\begin{verbatim}
$ useradd rserve -s /bin/false -U
\end{verbatim}

Now, find the user id and group id of the rserve user, and edit them in
rserve/Rserv.conf. This is so Rserve properly forks itself and runs as the
correct user when it is started.

\begin{verbatim}
$ id rserve
uid=1011(rserve) gid=1012(rserve) groups=1012(rserve)
\end{verbatim}

Next, we need to add the rserve user to the 'apache' group (The default
user for Tomcat), so it can communicate correctly with Tomcat and have the
necessary permissions for writing graphs. In this case, we will add apache
to the rserve group.

\begin{verbatim}
$ usermod -a -G rserve apache
\end{verbatim}

\section{Start Rserve}
Now we are ready to start Rserve! Run the script rserve/start.sh as ROOT
(or else Rserve will not fork itself properly). Then, check the rserve log
file (\emph{rserve.log}). The path to this log file can be changed by
modifying the script. Do a "ps -ef | grep rserve" to see if it has started.
Now, with Rserve installed and Postgres is running, Metrics is almost ready
to start generating some graphs!

\end{document}
